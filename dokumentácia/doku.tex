
\documentclass[a4paper, 11pt]{article}

\usepackage[czech]{babel}
\usepackage[utf8]{inputenc}
\usepackage[left=2cm, top=3cm, text={17cm, 24cm}]{geometry}
\usepackage{times}
\usepackage{graphicx}
\usepackage[unicode]{hyperref}
\hypersetup{
	colorlinks = true,
	hypertexnames = false,
	citecolor = black
}


\begin{document} %#################################################################################

%TITLEPAGE
\begin{titlepage}

\begin{center}
			\includegraphics[width=0.77\linewidth]{inc/FIT_logo.pdf} \\

			\vspace{\stretch{0.382}}
			

			\Huge{Projektová dokumentace} \\
			\vspace{0.5cm}
			\LARGE{\textbf{
				Varianta 3: COVID-19
			}} \\
			\vspace{0.5cm}
			\Large{Ukládání a příprava dat}

			\vspace{\stretch{0.618}}
		\end{center}

		{\Large
			
			\hfill \\
			Šimon Galba, Bc. (xgalba03) \\
			Gladiš Damián, Bc. (xgladi00) 	\\
			Jeřábek František, Bc. (xjerab25)\\

		}
					\vspace{1cm}
					{\Large
			\today}

\end{titlepage}	


	%%%%%%%%%%%%%%%%%%%%%%%%%%%%%%%% Obsah %%%%%%%%%%%%%%%%%%%%%%%%%%%%%%%%%%%%%
	\pagenumbering{roman}
	\setcounter{page}{1}
	\setlength{\parskip}{0pt}
{\hypersetup{hidelinks}\tableofcontents}
\setlength{\parskip}{0pt}
	\clearpage



	%%%%%%%%%%%%%%%%%%%%%%%%%%%%%%%% Úvod %%%%%%%%%%%%%%%%%%%%%%%%%%%%%%%%%%%%%%
	\pagenumbering{arabic}
	\setcounter{page}{1}

	\section{Úvod}

    Cieľom projektu je stiahnuť dáta štatistík o rôznych skutočnostiach súvisiacich s pandémiou COVID-19 za účelom ich ďalšieho spracovania a zobrazovania do grafov. Jedná sa prevažne o dáta zobrazujúce počty nakazených, vyliečených, novo hospitalizovaných, zaočkovaných, alebo počte vykonaných testov. \\ Tieto dáta sú spracovávané jazykom Python verzie 3.8 a ukladané do NON-SQL databázy MongoDB.  

    %%%%%%%%%%%%%%%%%%%%%%%%%%%%%%%% Spustenie programu %%%%%%%%%%%%%%%%%%%%%%
    \section{Spustenie}
    Keďže sa jedná o programovací jazyk Python, program sa bez prekladu spustí príkazom \textbf{python3 REST.py} 
    
    %%%%%%%%%%%%%%%%%%%%%%%%%%%%%%%% Knižnice %%%%%%%%%%%%%%%%%%%%%%
    \section{Využité knižnice}
    \begin{itemize}
        \item datetime - zmena formátu dátumu pre uloženie do databázy
        \item urlopen - sťahovanie dát z web-u
        \item ijson - stream-like parsovanie dát, využité na rýchlejšie zmeny dátumu dát
        \item pymongo.collection - prístup do databáze MongoDB
        \item MongoClient - prístup do databáze MongoDB
    \end{itemize}


	%%%%%%%%%%%%%%%%%%%%%%%%%%%%%%%% Návrh a implementace %%%%%%%%%%%%%%%%%%%%%%
	\section{Návrh a~implementácia}
    
    V hlavnom programe sa postupne vykonávajú funkcie na základe jednotlivých otázok na dáta. V jednotlivých funkciách vždy najprv prebehne stiahnutie dát a následná úprava formátu dátumu pre konzistentnosť. Potom sú dáta v dávkach pridávané do databázy. 
    
	 
\end{document}
